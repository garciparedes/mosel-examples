% !TEX root = ./article.tex

\documentclass[spanish]{article}

\usepackage{mystyle}
\usepackage{myvars}
\usepackage{mylinearprogramming}



%-----------------------------

\begin{document}

	\maketitle % Insert title

	\thispagestyle{fancy} % All pages have headers and footers


%-----------------------------
%	ABSTRACT
%-----------------------------

	\begin{abstract}
		\noindent En este documento se describe el problema de la \emph{planificación de tareas (scheduling)} para el caso concreto de una única máquina y minimización de la tardanza total ponderada como objetivo a minimizar. Para este problema se presentan distintas formulaciones exactas así como heurísticas de resolución aproximada. Además se presentan los resultados obtenidos mediante las distintas estrategias para 6 conjuntos de datos concretos. Por último se realiza una comparación de los resultados obtenidos por dichas estrategias tratando de describir las ventaajas y desventajas de cada una de ellas.
	\end{abstract}

%-----------------------------
%	TEXT
%-----------------------------

	\section{Introducción}
	\label{sec:intro}

		\paragraph{}
		El problema de la programación (o planificiación de tareas) consiste en la selección del orden de un conjunto de tareas o trabajos entre un número prefijado de máquinas o unidades de trabajo de tal manera que se minimice una determinada característica referida a las mismas. En este documento se trata el problema de la planificación de tareas referida a una única máquina, es decir, no se permite la paralelización de trabajo. La característica que se pretende minimizar es la tardanza total ponderada, la cual se define como la suma total de las unidades de tiempo de retraso de las tareas, cada una de las cuales tienen un peso (ponderación) diferente, por lo tanto, unas son más prioritarias que otras.

		\paragraph{}
		Para poder resolver este problema es necesario conocer un conjunto de valores referidos a cada tarea. Primero denotaremos cada tarea con el índice $j \in [1,n]$ siendo $n$ el número total de tareas a planificar. Para cada una de estas tareas se conoce $p_j$, que determina el tiempo de duración de la tarea $j$, $d_j$ que indica el momento de finalización máximo de la tarea $j$ y $w_j$ que se refiere al coste por unidad de tiempo de retraso de cada tarea.

		\paragraph{}
		Una vez conocidas dichas características y la medida que se pretende optimizar se está en condiciones de modelizar dicha expresión de manera matemática. Para ello es necesario declarar la variable $t_j$ que indica las unidades de tiempo de retraso de la tarea $j$. Por tanto, el objetivo del problema se muestra en la ecuación \eqref{eq:min-formula}.

		\begin{equation}
		\label{eq:min-formula}
			\text{Minimizar} \displaystyle\sum\limits_{i = 1}^{n} t_{j}w_{j}
		\end{equation}

		\paragraph{}
		A continuación se describen distintas formulaciones para resolver este problema. Las dos primeras (modelo disyuntivo y modelo de índices de tiempo) se corresponden con estrategias exactas mientras que las siguientes (heurística de selección aleatoria y heurística de búqueda local) se corresponden con soluciones aproximadas.

		\subsection{Modelo Disyuntivo}
		\label{sec:disyuntive}

			\paragraph{}
			El modelo disyuntivo se apoya en la utilización de las variables de decisión $x_{j}, s_{j}, r_{j}$ de carácter real positivo y las variables de decisión binarias $y_{ij}$. La idea es realizar todas las tareas evitando solapamientos entre ellas apoyandose en las variables continuas. La variable $x_{j}$ representa el momento de inicio de la tarea $j$, $s_{j}$ el momento de finalización de la misma y $r_{j}$ el retraso referido a la tarea $j$. Para representar el orden de las tareas se utiliza la matriz de variables binarias constituida por $y_{ij}$ de tal manera que la tarea $j$ se realiza en el momento $i$. La formulación completa se muestra en la ecuación \eqref{eq:disjunctive-formulation}.(La constante $M$ representa un número suficientemente grande.)

			\begin{eqfloat}
				\begin{equation}
					\begin{array}{ll@{}ll}
						\text{Minimizar}	& & \displaystyle\sum\limits_{j = 1}^{n} w_{j}r_{j} & \\
						\text{sujeto a}		& x_{j} + p_{j} + s_{j} - r_{j} 	&= d_{j}, 		& \forall j \in \{1,...,n\}\\
															&	x_{i} + p_{i} - x_{j} 					&\leq M(1-y_{ij}), 	& \forall i,j \in \{1,...,n\},  i < j\\
															&	x_{j} + p_{j} - x_{i} 					&\leq My_{ij}, 	& \forall i,j \in \{1,...,n\},  i < j\\
															&                               	&	x_{j}, s_{j}, r_{j} 	\geq 0, 	& \forall j \in \{1,...,n\} \\
															&                               				&	y_{ij} 	\in \{0,1\}, 	& \forall i,j \in \{1,...,n\},  i < j
					\end{array}
				\end{equation}
				\caption{Formulación Disyuntiva para el \emph{problema de programación de tareas (Scheduling)}.}
				\label{eq:disjunctive-formulation}
			\end{eqfloat}

		\subsection{Modelo de Índices de Tiempo}
		\label{sec:time-index}

			\paragraph{}
			La formulación de índices de tiempo se apoya en la utilización de un único tipo de variables de decisión de caracter binario denotadas como $x_{jt}$ donde $j$ se refiere a la tarea $j$ y $t$ al momento de inicio de dicha tarea donde $t \in \{1,2,...T\}$ y $T$ representa el tiempo máximo de finalización de todas las tareas, de decir, la suma de la duración de las mismas ($T = \sum_{j=1}^n p_j$). En este modelo la secuencia se restringe únicamente por las restricciones de orden, es decir, no se apoya en variables de holgura, por lo tanto presenta un gran número de restricciones y variables. Sin embargo, en la práctica esta formulación es mucho más eficiente que la descrita en la seccion anterior puesto que se resuelve mediante combinaciones binarias en lugar de ajustes de valores numéricos. La formulación de esta estrategia se muestra en la ecuación \eqref{eq:time-index-formulation}.

			\begin{eqfloat}
				\begin{equation}
					\begin{array}{ll@{}ll}
						\text{Minimizar}	& \displaystyle\sum\limits_{j = 1}^{n}\displaystyle\sum\limits_{t = 1}^{T} w_{j}x_{jt}max(\{0, t-1+p_{j}-d_{j}\}) && \\
						\text{sujeto a}		& \displaystyle\sum\limits_{t = 1}^{T - p_{j}+ 1} x_{j} &= 1, & \forall j \in \{1,...,n\}\\
															&	\displaystyle\sum\limits_{j = 1}^{n} \displaystyle\sum\limits_{s = max(\{0,t-p_{j}+1\})}^{t} x_{js} &\leq 1, 	& \forall t \in \{1,...,T\}\\
															&                               				&	x_{jt} 	\in \{0,1\}, 	& \forall j \in \{1,...,n\},  \forall t \in \{1,...,T\}
					\end{array}
				\end{equation}
				\caption{Formulación de Índices de tiempo para el \emph{problema de programación de tareas (Scheduling)}.}
				\label{eq:time-index-formulation}
			\end{eqfloat}

		\subsection{Heurística de Selección Aleatoria}
		\label{sec:random-heuristic}

			\paragraph{}
			La estrategia de selección aleatoria consiste en la generación de $N$ soluciones de manera aleatoria para seleccionar la mejor de entre todas ellas. Para ello se basa en al generación de números aleatorios de entre el conjunto de las $j$ tareas ($\{1,2,...n\}$) sin repetición. En cada iteración se compara la solución obtenida en la iteración anterior con la actual utilizando como medida la función objetivo descrita en la ecuación \eqref{eq:min-formula} para después seleccionar la que minimice dicha ecuación. Los resultados de esta estrategia suelen ser mucho peores que las de otras más sofisticadas, sin embargo su reducido coste computacional la convierte en una estrategia a considerar en algunos casos.

		\subsection{Heurística de Búsqueda Local}
		\label{sec:local-search-heuristic}

			\paragraph{}
			La heurística de búsqueda local se basa en la mejora de una solución inicial calculada previamente a partir de otra estrategia como selección aleatoria u otras heurísticas más sofisticadas para después tratar de mejorarla. La estrategia que utiliza para ello es la búsqueda del intercambio entre el orden de dos tareas que más minimice la ecuación \eqref{eq:min-formula} para después realizar fijar dicho intercambio y buscar un nuevo itercambio de índices a partir de dicha solución parcial. El algoritmo termina cuando no se encuentran nuevos intercambios que minimicen la ecuación \eqref{eq:min-formula}. Nótese que esta estrategia encontrará soluciones óptimas en aquellos casos en que no entre en mínimos locales de los cuales no pueda salir para encontrar el mínimo global.


	\section{Resolución de Problemas}
	\label{sec:problems}

		\paragraph{}
		En esta sección se presentan los resultados obtenidos tras resolver el \emph{problema de planificación de tareas} (Scheduling) con el objetivo de minimizar la tardanza ponderada sobre distintos conjuntos de datos de entrada. Dichos resultados se han agrupado por problema en lugar de por estrategia de resolución, lo cual permite comparar de manera más simple cada una de ellas. Las implementaciones se han llevado a cabo utilizando el entorno de desarrollo \emph{Xpress-Mosel} \cite{tool:xpress-mosel} utilizando el servidor \emph{Neos Server} \cite{tool:neos-server} para ejecutar las mismas. Puesto que las estrategias exactas presentan un elevado coste computacional se ha añadido la restricción de tiempo de cómputo de \emph{100 segundos}.

		\subsection{\texttt{sched\_10\_1}}

			\paragraph{}
			El conjunto de datos está formado por los tiempos de procesado $p_j$, los costes de retraso $w_j$ por unidad de tiempoy el momento máximo de terminación $d_j$ a partir de los cuales se debn planificar \textbf{10 tareas} de tal manera que se minimice la tardanza ponderada total, es decir, la suma de todos los costes de retraso. El problema ha sido resuelteo a partir de las estrategias descritas anteriormente (\emph{Modelo Disyuntivo} en la sección \ref{sec:disyuntive},\emph{Modelo de Índices de Tiempo} en la sección \ref{sec:time-index}, \emph{Heurística de Selección Aleatoria} con $100$ y $1000$ iteraciones en la sección \ref{sec:random-heuristic} y \emph{Heurística de Búsqueda Local} con $1$ y $4$ iteracciones en la sección \ref{sec:local-search-heuristic} )

			\begin{table}
				\centering
				\begin{tabu}{ | c | c | p{.58\linewidth} |}
					\hline
					\bfseries Modelo & \bfseries Retraso & \bfseries Programación
					\csvreader[head to column names]{../results/csv/sched_10_1.csv}{}
					{\\\hline\model&\delay&\schedule}
					\\\hline
				\end{tabu}
				\caption{Resultados Obtenidos para el problema \texttt{sched\_10\_1}}
				\label{table:sol-sched_10_1}
			\end{table}

			\paragraph{}
			Los resultados se muestran en la tabla \ref{table:sol-sched_10_1}. Para facilitar la comprensión a nivel de Tardanza Ponderada se ha incluido un diagrama de barras que muestra la relación entre las distintas estrategias. Dicho diagrama se muestra en la figura \ref{plot:sol-sched_10_1}.

			\begin{figure}
				\begin{center}
					\begin{tikzpicture}
						\begin{axis}[
							symbolic x coords={Disyuntivo (100 secs.), Indices de Tiempo (100 secs.),Seleccion aleatoria (100 iters),Seleccion aleatoria (1000 iters), Busqueda Local (1 iters), Busqueda Local (4 iters)},
							width=\textwidth,
							height=0.4\textwidth,
							xtick=data,
							ybar,
							ymajorgrids,
							bar width=1.25cm,
							x tick label style={text width=2.15cm}]
							\addplot table [x=model, y=delay, col sep=comma] {../results/csv/sched_10_1.csv};
						\end{axis}
					\end{tikzpicture}
				\end{center}
				\caption{Resultados Obtenidos para el problema \texttt{sched\_10\_1}}
				\label{plot:sol-sched_10_1}
			\end{figure}

		\subsection{\texttt{sched\_20\_1}}

			\paragraph{}
			El conjunto de datos está formado por los tiempos de procesado $p_j$, los costes de retraso $w_j$ por unidad de tiempoy el momento máximo de terminación $d_j$ a partir de los cuales se debn planificar \textbf{20 tareas} de tal manera que se minimice la tardanza ponderada total, es decir, la suma de todos los costes de retraso. El problema ha sido resuelteo a partir de las estrategias descritas anteriormente (\emph{Modelo Disyuntivo} en la sección \ref{sec:disyuntive},\emph{Modelo de Índices de Tiempo} en la sección \ref{sec:time-index}, \emph{Heurística de Selección Aleatoria} con $100$ y $1000$ iteraciones en la sección \ref{sec:random-heuristic} y \emph{Heurística de Búsqueda Local} con $1$ y $4$ iteracciones en la sección \ref{sec:local-search-heuristic} )


			\begin{table}
				\centering
				\begin{tabu}{ | c | c | p{.58\linewidth} |}
					\hline
					\bfseries Modelo & \bfseries Retraso & \bfseries Programación
					\csvreader[head to column names]{../results/csv/sched_20_1.csv}{}
					{\\\hline\model&\delay&\schedule}
					\\\hline
				\end{tabu}
				\caption{Resultados Obtenidos para el problema \texttt{sched\_20\_1}}
				\label{table:sol-sched_20_1}
			\end{table}

			\paragraph{}
			Los resultados se muestran en la tabla \ref{table:sol-sched_20_1}. Para facilitar la comprensión a nivel de Tardanza Ponderada se ha incluido un diagrama de barras que muestra la relación entre las distintas estrategias. Dicho diagrama se muestra en la figura \ref{plot:sol-sched_20_1}.

			\begin{figure}
				\begin{center}
					\begin{tikzpicture}
						\begin{axis}[
							symbolic x coords={Disyuntivo (100 secs.), Indices de Tiempo (100 secs.),Seleccion aleatoria (100 iters),Seleccion aleatoria (1000 iters), Busqueda Local (1 iters), Busqueda Local (4 iters)},
							width=\textwidth,
							height=0.4\textwidth,
							xtick=data,
							ybar,
							ymajorgrids,
							bar width=1.25cm,
							x tick label style={text width=2.15cm}]
							\addplot table [x=model, y=delay, col sep=comma] {../results/csv/sched_20_1.csv};
						\end{axis}
					\end{tikzpicture}
				\end{center}
				\caption{Resultados Obtenidos para el problema \texttt{sched\_20\_1}}
				\label{plot:sol-sched_20_1}
			\end{figure}

		\subsection{\texttt{sched\_30\_1}}

			\paragraph{}
			El conjunto de datos está formado por los tiempos de procesado $p_j$, los costes de retraso $w_j$ por unidad de tiempoy el momento máximo de terminación $d_j$ a partir de los cuales se debn planificar \textbf{30 tareas} de tal manera que se minimice la tardanza ponderada total, es decir, la suma de todos los costes de retraso. El problema ha sido resuelteo a partir de las estrategias descritas anteriormente (\emph{Modelo Disyuntivo} en la sección \ref{sec:disyuntive},\emph{Modelo de Índices de Tiempo} en la sección \ref{sec:time-index}, \emph{Heurística de Selección Aleatoria} con $100$ y $1000$ iteraciones en la sección \ref{sec:random-heuristic} y \emph{Heurística de Búsqueda Local} con $1$ y $4$ iteracciones en la sección \ref{sec:local-search-heuristic} )


			\begin{table}
				\centering
				\begin{tabu}{ | c | c | p{.58\linewidth} |}
					\hline
					\bfseries Modelo & \bfseries Retraso & \bfseries Programación
					\csvreader[head to column names]{../results/csv/sched_30_1.csv}{}
					{\\\hline\model&\delay&\schedule}
					\\\hline
				\end{tabu}
				\caption{Resultados Obtenidos para el problema \texttt{sched\_30\_1}}
				\label{table:sol-sched_30_1}
			\end{table}

			\paragraph{}
			Los resultados se muestran en la tabla \ref{table:sol-sched_30_1}. Para facilitar la comprensión a nivel de Tardanza Ponderada se ha incluido un diagrama de barras que muestra la relación entre las distintas estrategias. Dicho diagrama se muestra en la figura \ref{plot:sol-sched_30_1}.


			\begin{figure}
				\begin{center}
					\begin{tikzpicture}
						\begin{axis}[
							symbolic x coords={Disyuntivo (100 secs.), Indices de Tiempo (100 secs.),Seleccion aleatoria (100 iters),Seleccion aleatoria (1000 iters), Busqueda Local (1 iters), Busqueda Local (4 iters)},
							width=\textwidth,
							height=0.4\textwidth,
							xtick=data,
							ybar,
							ymajorgrids,
							bar width=1.25cm,
							x tick label style={text width=2.15cm}]
							\addplot table [x=model, y=delay, col sep=comma] {../results/csv/sched_30_1.csv};
						\end{axis}
					\end{tikzpicture}
				\end{center}
				\caption{Resultados Obtenidos para el problema \texttt{sched\_30\_1}}
				\label{plot:sol-sched_30_1}
			\end{figure}

		\subsection{\texttt{sched\_40\_1}}

			\paragraph{}
			El conjunto de datos está formado por los tiempos de procesado $p_j$, los costes de retraso $w_j$ por unidad de tiempoy el momento máximo de terminación $d_j$ a partir de los cuales se debn planificar \textbf{40 tareas} de tal manera que se minimice la tardanza ponderada total, es decir, la suma de todos los costes de retraso. El problema ha sido resuelteo a partir de las estrategias descritas anteriormente (\emph{Modelo Disyuntivo} en la sección \ref{sec:disyuntive},\emph{Modelo de Índices de Tiempo} en la sección \ref{sec:time-index}, \emph{Heurística de Selección Aleatoria} con $100$ y $1000$ iteraciones en la sección \ref{sec:random-heuristic} y \emph{Heurística de Búsqueda Local} con $1$ y $4$ iteracciones en la sección \ref{sec:local-search-heuristic} )

			\begin{table}
				\centering
				\begin{tabu}{ | c | c | p{.58\linewidth} |}
					\hline
					\bfseries Modelo & \bfseries Retraso & \bfseries Programación
					\csvreader[head to column names]{../results/csv/sched_40_1.csv}{}
					{\\\hline\model&\delay&\schedule}
					\\\hline
				\end{tabu}
				\caption{Resultados Obtenidos para el problema \texttt{sched\_40\_1}}
				\label{table:sol-sched_40_1}
			\end{table}

			\paragraph{}
			Los resultados se muestran en la tabla \ref{table:sol-sched_40_1}. Para facilitar la comprensión a nivel de Tardanza Ponderada se ha incluido un diagrama de barras que muestra la relación entre las distintas estrategias. Dicho diagrama se muestra en la figura \ref{plot:sol-sched_40_1}.


			\begin{figure}
				\begin{center}
					\begin{tikzpicture}
						\begin{axis}[
							symbolic x coords={Disyuntivo (100 secs.), Indices de Tiempo (100 secs.),Seleccion aleatoria (100 iters),Seleccion aleatoria (1000 iters), Busqueda Local (1 iters), Busqueda Local (4 iters)},
							width=\textwidth,
							height=0.4\textwidth,
							xtick=data,
							ybar,
							ymajorgrids,
							bar width=1.25cm,
							x tick label style={text width=2.15cm}]
							\addplot table [x=model, y=delay, col sep=comma] {../results/csv/sched_40_1.csv};
						\end{axis}
					\end{tikzpicture}
				\end{center}
				\caption{Resultados Obtenidos para el problema \texttt{sched\_40\_1}}
				\label{plot:sol-sched_40_1}
			\end{figure}

		\subsection{\texttt{sched\_100\_1}}

			\paragraph{}
			El conjunto de datos está formado por los tiempos de procesado $p_j$, los costes de retraso $w_j$ por unidad de tiempoy el momento máximo de terminación $d_j$ a partir de los cuales se debn planificar \textbf{100 tareas} de tal manera que se minimice la tardanza ponderada total, es decir, la suma de todos los costes de retraso. El problema ha sido resuelteo a partir de las estrategias descritas anteriormente (\emph{Modelo Disyuntivo} en la sección \ref{sec:disyuntive},\emph{Modelo de Índices de Tiempo} en la sección \ref{sec:time-index}, \emph{Heurística de Selección Aleatoria} con $100$ y $1000$ iteraciones en la sección \ref{sec:random-heuristic} y \emph{Heurística de Búsqueda Local} con $1$ y $4$ iteracciones en la sección \ref{sec:local-search-heuristic} )


			\begin{table}
				\centering
				\begin{tabu}{ | c | c | p{.58\linewidth} |}
					\hline
					\bfseries Modelo & \bfseries Retraso & \bfseries Programación
					\csvreader[head to column names]{../results/csv/sched_100_1.csv}{}
					{\\\hline\model&\delay&\schedule}
					\\\hline
				\end{tabu}
				\caption{Resultados Obtenidos para el problema \texttt{sched\_100\_1}}
				\label{table:sol-sched_100_1}
			\end{table}

			\paragraph{}
			Los resultados se muestran en la tabla \ref{table:sol-sched_100_1}. Para facilitar la comprensión a nivel de Tardanza Ponderada se ha incluido un diagrama de barras que muestra la relación entre las distintas estrategias. Dicho diagrama se muestra en la figura \ref{plot:sol-sched_100_1}.


			\begin{figure}
				\begin{center}
					\begin{tikzpicture}
						\begin{axis}[
							symbolic x coords={Disyuntivo (100 secs.), Indices de Tiempo (100 secs.),Seleccion aleatoria (100 iters),Seleccion aleatoria (1000 iters), Busqueda Local (1 iters), Busqueda Local (4 iters)},
							width=\textwidth,
							height=0.4\textwidth,
							xtick=data,
							ybar,
							ymajorgrids,
							bar width=1.25cm,
							x tick label style={text width=2.15cm}]
							\addplot table [x=model, y=delay, col sep=comma] {../results/csv/sched_100_1.csv};
						\end{axis}
					\end{tikzpicture}
				\end{center}
				\caption{Resultados Obtenidos para el problema \texttt{sched\_100\_1}}
				\label{plot:sol-sched_100_1}
			\end{figure}

		\subsection{\texttt{sched\_200\_1}}

			\paragraph{}
			El conjunto de datos está formado por los tiempos de procesado $p_j$, los costes de retraso $w_j$ por unidad de tiempoy el momento máximo de terminación $d_j$ a partir de los cuales se debn planificar \textbf{200 tareas} de tal manera que se minimice la tardanza ponderada total, es decir, la suma de todos los costes de retraso. El problema ha sido resuelteo a partir de las estrategias descritas anteriormente (\emph{Modelo Disyuntivo} en la sección \ref{sec:disyuntive},\emph{Modelo de Índices de Tiempo} en la sección \ref{sec:time-index}, \emph{Heurística de Selección Aleatoria} con $100$ y $1000$ iteraciones en la sección \ref{sec:random-heuristic} y \emph{Heurística de Búsqueda Local} con $1$ y $4$ iteracciones en la sección \ref{sec:local-search-heuristic} )


			\csvreader[
			  longtable= | c | c | p{.58\linewidth} |,
			  table head=\hline \bfseries Modelo & \bfseries Retraso & \bfseries Programación \\ \hline \endhead,
  			table foot= \caption{Resultados Obtenidos para el problema \texttt{sched\_200\_1}}\label{table:sol-sched_200_1}\\,
				late after line=\\\hline,
  			before reading={\catcode`\#=12},after reading={\catcode`\#=6}
			]{../results/csv/sched_200_1.csv}{1=\model,2=\delay,3=\schedule}{\model&\delay&\schedule}

			\paragraph{}
			Los resultados se muestran en la tabla \ref{table:sol-sched_200_1}. Para facilitar la comprensión a nivel de Tardanza Ponderada se ha incluido un diagrama de barras que muestra la relación entre las distintas estrategias. Dicho diagrama se muestra en la figura \ref{plot:sol-sched_200_1}.

			\begin{figure}
				\begin{center}
					\begin{tikzpicture}
						\begin{axis}[
							symbolic x coords={Disyuntivo (100 secs.), Indices de Tiempo (100 secs.),Seleccion aleatoria (100 iters),Seleccion aleatoria (1000 iters), Busqueda Local (1 iters), Busqueda Local (4 iters)},
							width=\textwidth,
							height=0.4\textwidth,
							xtick=data,
							ybar,
							ymajorgrids,
							bar width=1.25cm,
							x tick label style={text width=2.15cm}]
							\addplot table [x=model, y=delay, col sep=comma] {../results/csv/sched_200_1.csv};
						\end{axis}
					\end{tikzpicture}
				\end{center}
				\caption{Resultados Obtenidos para el problema \texttt{sched\_200\_1}}
				\label{plot:sol-sched_200_1}
			\end{figure}

	\section{Conclusiones}

		\paragraph{}
		Tal y como se puede apreciar en los resultados que se muestran en la sección anterior, las distintas estrategias presentan una clara tendencia. En el caso de las estrategias exactas (y teniendo en cuenta la restricción fijada a priori de tiempo de cómputo de 100 segundos), la formulación de índices de tiempo encuentra la solución en tiempos mucho más reducidos (en problemas grandes la formulación disyuntiva no consigue encontrar soluciones aceptables).

		\paragraph{}
		En cuanto a las estrategias basadas en heurísticas, la selección aleatoria presenta malos resultados en todos los casos (a pesar de ello es la más simple a nivel conceptual) aunque se puede apreciar una clara tendencia: cuando se aumenta el número de iteraciones los resultados mejoran, lo cual es un suceso esperable si el generador de números aleatorios genera los valores siguiendo una distribución uniforme. La estrategia de búsqueda local por contra, presenta resultados muy cercanos al óptimo (en un gran número de ocasiones el óptimo) a partir de una única iteración (que cuando se repite 4 iteraciones siempre encuentra el óptimo para los conjuntos de datos utilizados). Por contra, esta estrategia presenta un elevado coste computacional con respecto del tamaño del problema, el cual deja de ser asumible para problemas con un elevado número de tareas.

		\paragraph{}
		En cuanto a la comparación entre las estrategias exactas (teniendo en cuenta la restricción de tiempo de cómputo) y las heurísticas, no es posible llevar a cabo una comparación apropiada puesto que los resultados de las heurísticas no han sido calculados teniendo en cuenta restricciones de tiempo. Sin embargo, para problemas de tamaño reducido en los cuales ambos tipos de estrategias terminan antes del tiempo acotado de 100 segundos, la estrategia basada en búsqueda local presenta soluciones muy cercanas al óptimo encontrado por el modelo de índices de tiempo utilizando tiempos de cómputo mucho menores.

%-----------------------------
%	BIBLIOGRAPHY
%-----------------------------
	\nocite{subject:mio}
	\nocite{garciparedes:mosel-examples}
	\nocite{tool:xpress-mosel}
	\nocite{tool:neos-server}
	\bibliographystyle{alpha}
  \bibliography{bib/misc}

\end{document}
